\documentclass[11pt]{article}
\usepackage{amsmath}
\usepackage{amsfonts}
\usepackage{dirtytalk}
\usepackage{mathtools}
\usepackage{bm}
\usepackage{enumitem}
\usepackage[dvipsnames]{xcolor}
\usepackage{minted}
\usepackage{parskip}
\usepackage{pythonhighlight}
\usepackage{graphicx}
\usepackage{hyperref}
\hypersetup{
    colorlinks=true,
    linkcolor=blue,
    filecolor=magenta,      
    urlcolor=cyan,
    pdftitle={Overleaf Example},
    pdfpagemode=FullScreen,
    }
\graphicspath{
  {../../}
}
\usepackage[margin=72pt]{geometry}
\setlength{\parindent}{0pt}
\definecolor{puccin}{HTML}{24273a}
\definecolor{bgrd}{HTML}{212228}
\pagecolor{bgrd}
\color{white}

\title{How good does your coin need to be to beat the market?}
\begin{document}
\maketitle

The other day I was making a really simple classifier that would predict if the S\&P would close higher than it opened.
It was getting about 58\% correct in live testing which seemed pretty good but I dont have any benchmark for whats \emph{good} or \emph{bad}. This lead to the following question: If every mornign we flipped a coin where heads means we buy and sell at close, and tails means we do nothing, how accurate does that coin need to be to beat the market?

More formally lets say $X$ is our coin with probability $q$ and skill $\alpha$, and the underlying stock is $S_t$. The returns from the coin strategy $C_t$ and the returns from simply buying and holding $R_t$ would be
\begin{equation*}
    C_t = \sum_{i=t_0}^{t} q\alpha (S_{t+1} - S_{t})
\end{equation*}
\begin{equation*}
    R_t = S_{t} - S_{t_0}
\end{equation*}
What I want to know is under what conditions is $\mathbb{E}[C_t] > \mathbb{E}[R_t]$
\end{document}
