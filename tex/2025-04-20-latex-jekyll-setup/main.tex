
\documentclass[11pt]{article}
\usepackage{amsmath}
\usepackage{amsfonts}
\usepackage{dirtytalk}
\usepackage{mathtools}
\usepackage{bm}
\usepackage{enumitem}
\usepackage[dvipsnames]{xcolor}
\usepackage{minted}
\usepackage{parskip}
\usepackage{pythonhighlight}
\usepackage{graphicx}
\usepackage{hyperref}
\hypersetup{
    colorlinks=true,
    linkcolor=blue,
    filecolor=magenta,      
    urlcolor=cyan,
    pdftitle={Overleaf Example},
    pdfpagemode=FullScreen,
    }
\graphicspath{
  {../../}
}
\usepackage[margin=72pt]{geometry}
\setlength{\parindent}{0pt}
\definecolor{puccin}{HTML}{24273a}
\definecolor{bgrd}{HTML}{212228}
\pagecolor{bgrd}
\color{white}

\setminted[python]{
  breaklines = true,
  bgcolor = puccin,
  style = monokai,
  fontsize = \footnotesize
}
\title{LaTeX Jekyll Blog: Trials and Tribulations}
\begin{document}
\maketitle
All I wanted was a simple way to type up my notes/musings in LaTeX, but it took a suprising emount of effort to get it set up.
I am admittedly pretty stupid, so most of my issues here were probably user error, but it took me the better part of a Saturday
to get this set up properly.
\\\\
The workflow I settled for is as follows
\begin{enumerate}
    \item write all my \texttt{.tex} files in neovim as I would normally
    \item Use pandoc to convert to \texttt{.tex} $\rightarrow$ \texttt{.html}
    \item Use a python script to remove pandoc-generated \texttt{.css} from the \texttt{.html}
    \item deploy the whole thing with jekyll
\end{enumerate}
Its not that complicated but it took a lot of experimentation to figure this out. The code is in the \href{https://github.com/sammy-snipes/sammy-snipes.github.io}{repo hosting this site}. In particular I struggled finding a good way to convert \texttt{.tex}  $\rightarrow$ \texttt{.html}. Theres too many options: LaTeXML, TeX4ht, Pandoc, plasTeX...etc. I tried using make4ht at first which rendered all the equations as images, then I needed a script to put them all in the right folders when building... it got complicated. 
\\\\ 
But the pandoc solution is pretty good. Im converting with pandoc in standalone mode and theres
just two small issues. The first is an annoying \texttt{<!DOCTYPE...} string that it sticks
at the top, and then gets rendered in the site, but its simple enough to remove that. 
The second is in standalone mode it generates its own \texttt{.css} which is hard to 
override later (it might not be hard, but im an idiot. I dont know how \texttt{html} or \texttt{.css} work). So with a simple python script we can make it work 

\begin{minted}{python}

import re
import sys

def clean_html(html_content):
    html_content = re.sub(
        r"<!DOCTYPE[^>]*>\s*", "", html_content, flags=re.IGNORECASE
    )
    html_content = re.sub(
        r"<style.*?>.*?</style>",
        "",
        html_content,
        flags=re.DOTALL | re.IGNORECASE,
    )
    return html_content


if __name__ == "__main__":
    input_file = sys.argv[1]
    output_file = sys.argv[2]

    with open(input_file, "r", encoding="utf-8") as f:
        html = f.read()

    cleaned_html = clean_html(html)

    with open(output_file, "w", encoding="utf-8") as f:
        f.write(cleaned_html)

\end{minted}
this deletes the annoying string, and deletes the css. This way the cite css is applied to the pandoc generated \texttt{html}.

\end{document}
